\documentclass[a4paper]{article}
\pagestyle{empty}
\sqldb{testsqltex}
\begin{document}

\sqlfield[setvar=1]{SELECT cust_nr FROM invoice WHERE nr = $PAR1}

\noindent\textbf{\sqlrow[fldsep=\\]{
SELECT name 
,      city
FROM   customer
WHERE  nr = '$VAR1'
}}

\vspace{8mm}

\sqlfield[setvar=2]{SELECT SUM(l.amount * a.price)
FROM   invoice      i
,      invoice_line l
,      article      a
WHERE  i.nr         = $PAR1
AND    l.invoice_nr = i.nr
AND    a.nr         = l.article_nr
}

\begin{tabbing}
Invoice number: \= \sqlfield{SELECT nr FROM invoice WHERE nr = $PAR1} \\
Invoice date:   \> \sqlfield{SELECT invoice_date FROM invoice WHERE nr = $PAR1} \\
Total amount:   \> EUR \sqlfield{SELECT FORMAT($VAR2,2) FROM invoice WHERE nr = $PAR1} \\
Due date:       \> \sqlfield{SELECT due_date FROM invoice WHERE nr = $PAR1} \\
\end{tabbing}

\begin{tabular}{lp{30mm}ll|l}\\ \hline
{\bfseries Nr} &
{\bfseries Article} &
{\bfseries Amount} &
{\bfseries Price} &
{\bfseries Total} \\ \hline\hline
\sqlrow[fldsep= &,rowsep= \\]{
SELECT l.nr
,      a.name
,      l.amount
,      CONCAT("EUR ",a.price)
,      CONCAT("EUR ",FORMAT(l.amount*a.price,2))
FROM   invoice      i
,      invoice_line l
,      article      a
WHERE  i.nr          = $PAR1
AND    l.invoice_nr  = i.nr
AND    a.nr          = l.article_nr
}\\\hline
 & & & & \textbf{EUR \sqlfield{SELECT FORMAT($VAR2,2)}} \\
\end{tabular}

\end{document}
